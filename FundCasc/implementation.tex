\section{Implementation}
\label{sec:impl}

We have implemented a prototype of LD-Sketch.
In our implementation, each remote site and worker is a process
and they are connected with TCP connections or in-memory ring buffers.
In this section, we will discuss some implementation issues.

\subsection{Optimize I/O Communications}

\begin{algorithm}[t]
\caption{I/O Efficient Communication Mechanism}
\label{alg:io}
\begin{small}
\begin{algorithmic}[1]
\item[{\bf Input}: data item $(x, v_x)$]
\Function {Emit}{$x,v_x$}
%  \State $StepSet = \{i: gcd(q, i) = 1\}$
%  \State $base = HashFunc1(x) \% q$
%  \State $step = HashFunc2(x) \% StepSet.size$
%  \State $index = HashFun3(x, x.timestamp) \% d$
%  \State $target = (base + step * index) \% q$
  \State $t' = $ current time
  \State randomly select $d$ workers based on $x$
  \State randomly select a target among the $d$ based on $x$ and $t'$
  \State send $(x, v_x)$ to the target worker
\EndFunction
\State
\Procedure {EfficientEmit}{}
  \State $C \Leftarrow \emptyset$
  \For {data item $(x,v_x)$}
    \If{$x \in C$}
      \State $C[x] = C[x] + v_x$
    \Else
      \State generate a random value $0<p_{r}\le 1$
      \If{$p_{r}<\hat{p}$ and $C$ is not full}
        \State $C[x] = v_x$
      \Else
        \State {\sc Emit}($x, v_x$)
      \EndIf
    \EndIf
  \EndFor
  \For {$x \in C$}
    \State {\sc Emit}($x, C[x]$)
  \EndFor
\EndProcedure
\end{algorithmic}
\end{small}
\end{algorithm}

We propose a mechanism to efficient I/O communication.
We use number of I/O emitted events as the metric.
In Section \ref{sec:distributed}, the remote sites emit collected data items to workers directly.
The number of I/O communication is the same as the number of data items.

The efficient communication of LD-Sketch employs a small size cache $C$ in each remote site.
With the cache, data items of some keys can be aggregated and sent in a batched emitted event.
The challenge is how to select keys.
Intuitively, it brings more benefits to store a key with a large amount of data items than a key with less data items.
LD-Sketch caches each data item with a probability $\hat{p}$, which is selected as a small value.
Thus, the more data items a key has, the higher probability it would be cached.

Algorithm~\ref{alg:io} outlines the efficient communication mechanism of LD-Sketch.
The function {\sc Emit} (Lines 1-6) is the d-choices approach in Section \ref{sec:distributed},
and we use it as a building block for the efficient communication mechanism.
Initially, the cache is empty (Line 9).
On the arrival of a data item, if its key is already cached, LD-Sketch updates the aggregated sum in the cache (Lines 11-12).
Otherwise, LD-Sketch caches is with probability $\hat{p}$ (Lines 14-16).
If the data item is not cached, it will be sent to one of target workers with the d-choices approach (Line 18).
At the end of the time epoch, all the aggregated data items in the cache will be sent (Lines 22-24).

We first analysis the number of communications with the mechanism.
For a key with $\alpha$ data items, it will be cached at its $i$-th data item with probability $\hat{p}(1-\hat{p})^{i-1}$.
The expected number of I/O communication is reduced to
$(1-\hat{p})^{\alpha}*\alpha + \sum_{i=1}^{\alpha}i(1-\hat{p})^{i-1}\hat{p} = \frac{1-(1-\hat{p})^{\alpha}}{\hat{p}}$.
%\begin{lemma}
%  Given $\hat{p}>0$, for each $x$ with $U_t^i(x)$ updates at input stream $i$, the expected number I/O communication is $\frac{1-(1-\hat{p})^{U_t^i(x)}}{\hat{p}} < \frac{1}{\hat{p}}$.
%\end{lemma}
%
%  \begin{theorem}
%      Given a small cache storing up to $\Delta$ keys, the I/O communication can be reduced to $O(C_t(i))$.
%  \end{theorem}

The caching technique is quite similar to the Sample-and-Hold algorithm in the work \cite{Estan2003}.
The difference is that they drop the uncached data items, while LD-Sketch emits them to the workers
so that no additional communication error will be introduced.
The expected memory usage and the standard deviation of the efficient communication mechanism is $\hat{p}\hat{U}$ and $\sqrt{\hat{U}\hat{p}(1-\hat{p})}$,
respectively.
Here $\hat{U}$ is the total amount of data items in the epoch.
The detailed analysis can be found in \cite{Estan2003}.

In the context of network traffic, the mechanism can be further improved.
Our implementation considers that most keys are short-term in practise.
When a key in the cache is not updated for long time, it will be emitted and the entry in the cache will be released.

%\subsection{Parameter Selection}
%
%In Section \ref{sec:local}, we give an approach to select $r, w$ and $T$ if $\phi, \epsilon$ and $\delta$ are given.
%Now we propose an approach to determine $w$ when the memory size is fixed.
%By Lemma \ref{lemma:bucket_length}, $O(r(w+\frac{U^2}{wT^2}))$