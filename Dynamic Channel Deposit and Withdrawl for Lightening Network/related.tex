\section{Related Work}
\label{sec:related}

%We review related work in heavy hitter/changer detection.

{\bf Counter-based techniques.}
The Misra-Gries algorithm \cite{Misra1982} maintains an associative array of
counters and keeps tracks of frequent items (see
Section~\ref{subsec:background}).
%and guarantees that all heavy hitters are kept in the array.
%algorithm \cite{Moore81} to find out all items with frequency larger than
%some threshold.  It updates the counters for each arriving data item by a set
%of rules, and guarantees that all heavy hitters are kept in the counters.
%$n/k$, where $k$ is an integer and $n$ is the total count.
%The algorithm maintains $k-1$ counters initialized with 0.
%The algorithm guarantees that all keys exceeding $n/k$ are kept in the
%counters.  Furthermore, the counters estimate the frequencies with error at
%most $n/k$.
%This approach can be extended for deterministically identifying heavy hitters.
%Bose et al. \cite{Bose2003} select $1/\epsilon$ as $k$ for the MisraGries algorithm, resulting in the estimation error at most $\epsilon n$.
Lossy Counting \cite{Manku2002} extends the Misra-Gries algorithm by tracking
the estimation error for each key.  Space-Saving \cite{Metwally2005} improves
the time and space efficiency with some specialized data structure.
%``Stream-Summary''.  
Probabilistic Lossy Counting \cite{Dimitropoulos2008}
provides probabilistic guarantees on accuracy and consumes less memory by
allowing errors in Lossy Counting.  The above algorithms only work for heavy
hitter detection, but do not address heavy changer detection.  
%In addition, they assume the arriving data items with equal weights, while
%our work considers weighted inputs.

{\bf Sketch-based techniques.} 
%A sketch randomly projects a large set into a significantly small summary and
%operates on the projected summary.
Count-Sketch \cite{Charikar2004,Cormode2012} and Count-Min \cite{Cormode2005a}
are two well-known sketch algorithms for frequency estimation.
%The original version of Count-Min only applies to heavy hitter detection, and
%a modification \cite{Cormode2012} makes it feasible for heavy changer
%detection by selecting the median instead of the minimum value of sketch rows
%as estimation.  
The two algorithms differ in space requirement and error bounds.
%Count-Sketch bounds the error with the $L_2$ norm instead of the $L_1$ norm of
%Count-Min, while increasing the space requirement by a factor of $1/\epsilon$.
A comprehensive study \cite{Cormode2010} shows that both algorithms have
similar performance in practice.  Sketches have been used in heavy hitter
detection (e.g., \cite{Estan2003}) and heavy changer detection (e.g.,
\cite{Krishnamurthy2003}).  
%However, the main drawback of sketches is that
%they cannot directly restore which heavy keys are kept.  One must enumerate
%all possible keys and query the estimated volume, and it is computationally
%infeasible due to the huge key space.

Some studies propose to efficiently restore heavy keys.
%The key idea is to maintain extra data structures when updating the sketches,
%on which the recovery process is based.
%Some studies are based on pruning the search space.  
Cormode et al.  \cite{Cormode2003} consider a hierarchical structure of keys
and formalize the problem of finding hierarchical heavy hitters.
%They search the hierarchical key space by a top-down approach.
%Such approaches produce accurate results in heavy hitter detection, but for
%heavy changers, the positive and negative changes in the same subspace may
%cancel each other and result in false negatives.
%Furthermore, searching (even the subset of) key space still consumes a lot of
%computational resources.
%Other studies aim to decode keys directly instead of searching the key space.
Deltoids \cite{Cormode2004} exploits group testing to construct heavy keys
from the extra information of buckets.  Reversible Sketch \cite{Schweller2007}
generalizes this approach by searching smaller divided keys and constructing
heavy keys from the searched results.  SeqHash \cite{Bu2010} constructs a
sequence of subspaces and searches each subspace based on the results of prior
ones.  Fast Sketch \cite{Liu2012} optimizes the space and time complexity of
group testing by a quotient technique.
%The drawback of the above construction-based approaches is that the heavy key
%cannot be always recovered even though we know there exists heavy keys there.
Group testing assumes that there is exactly one heavy key in a bucket.
%If there are multiple heavy keys, none of them will be recovered.
It leads to a high false negative rate when multiple heavy keys are hashed to
the same bucket due to constrained memory, as shown in our evaluation.  In
contrast, LD-Sketch queries
%the estimation volume for top
the heavy key candidates in the buckets and guarantees no false negatives.
%Moreover, it achieves high detection performance by searching only a small
%number of heavy key candidates. 

%{\bf Denoising techniques.}  Many studies focus on reducing noises due to
%collision of keys.  The multi-stage filter \cite{Estan2003} regards sketches
%as parallel filters to reduce false positives.  It proposes a conservative
%update technique to mitigate noises.  The spectral bloom filter
%\cite{Cohen2003} employs a similar conservative update technique, and uses a
%secondary sketch to filter out false positives.  Chen et al.  \cite{Chen2010}
%consider long duration flows in network traffic as heavy hitters, and use two
%sketches for adjacent time intervals to filter noises in each time interval.
%%The sketch corresponding to current time interval is conservatively updated
%%based on the other one corresponding to the last time interval.
%The above studies maintain exactly one counter in each bucket and aim to
%improve the accuracy of the counter.  Our work differs from them by employing
%a dynamic dictionary that stores the top-ranked keys for each bucket.

{\bf Distributed detection.}
%The linear property of sketch makes it work well in distributed environment.
%The typically setting is to
Cormode et al. \cite{Cormode2005b} exploit the linear property of sketches
for distributed detection, 
%.  Each remote site maintains a sketch, while a centralized worker merges the
%sketches and performs detection.  They also
and reduce I/O using a prediction model. 
%both remote sites and the centralized node.  The centralized node perform
%detection based on the predicted sketches, while the remote sites maintain
%the truly sketch as well as the predicted sketch.  A remote site sends its
%sketch to the centralized node only when the difference between its predicted
%sketch and truly sketch is larger than a threshold.
%One drawback of the approach is that the remote sites must maintain a sketch
%of the same size as that in the centralized worker, and the sketch size can be
%huge.
%Moreover, the CPU-intensive detection may become a bottleneck in the
%centralized node.
Manjhi et al. \cite{Manjhi2005}
%take the load of workers into account.  They
organize all workers into a tree structure, in which each level has its own
error parameter to control the detection accuracy.
%with the remote sites as leaves and the centralized node as root.
%Each level in the tree has its own error parameter.
%Nodes detect heavy hitters locally with the error parameter in their level
%and send the detected results to their parent nodes.
%By changing the error parameters, this work achieves different goals, such as
%the communication cost in the link or the maximum load in the root node.
%The tree design, however, may fail to work when the key distribution is
%unbalanced among workers.
%imbalanced among the remote sites.
Yi et al. \cite{Yi2009} address the worst-case communication complexity in
distributed streaming.
%allow workers to send feedback messages to remote workers.
%They propose an algorithm reaching the lower bound of communication costs in
%this scenario.
%The above studies trade between accuracy and communication efficiency.
However, the above studies only address heavy hitter detection, while the
distributed detection of LD-Sketch addresses heavy changer detection as well. 
%Thus, each communication channel may introduce an additional error, which may
%prevent the distributed architecture from scaling to large.  As a comparison,
%our work optimizes I/O communication without introducing any %communication
%errors. Recently, Yu et al. \cite{Yu2013} deploy heavy key detection
%algorithms in software-defined networks \cite{Yu2013}.
%programmable networking devices \cite{Jose2011} and perform 
%These approaches highly depend on the underlying frameworks in deployment. 
%and generalizing them for different frameworks remains an open issue.
%TCAM \cite{Jose2011} is a framework for distributed flow counting.
%They customize the hierarchical heavy hitter detection algorithm \cite{Cormode2003} atop the framework.
%OpenSketch \cite{Yu2013} deploys sketches in routers and builds heavy hitter detection with the sketches.
%It remains an open issue whether they can be generalized in the scenarios without such frameworks.
