\section{Introduction}
\label{sec:introduction}


Existing blockchain technology is held back by its serious performance and scalability issues, such as low transaction capacity and prolonged delay. While Bitcoin\cite{nakamoto2008bitcoin} supports less than seven transactions per second, Visa, on the contrary, supports up to 47,000 transactions per second at peak processing power. In Bitcoin, a block is generated every 10 minutes and to ensure the validity of transactions, it requires 6 confirmations, which will take approximately 60 minutes to complete. Bitcoin’s inefficiency and poor scalability inevitably hinder the usability of blockchain in areas such as micropayments.
 
Lightening Network (LN)\cite{poon2015bitcoin} is a practical approach to improve transaction capacity and reduce transaction delay using a network of off-chain micropayment channels\cite{DBLP:journals/corr/LindEPS16}. In LN, any two participants can collectively build a bidirectional off-chain channel by depositing funds into the channel. They can later agree on any new allocation regarding the funds which are already committed to the channel. With a cryptographic technique namely RSMC (Revocable Sequence Maturity Contract), LN guarantees that neither participant has any incentives to deny the latest agreement unless they broadcast their agreement to the main chain. Whomever violates the terms first will lose all of his/her personal funds. In addition to RSMC that ensures unconditional transfer of fundings, LN proposes another mechanism, HTLC (Hashed Timelock Contract), which can be used for conditional transfer. The combination of RSMC and HTLC, together with a large network of micropayment channels on the main blockchain of the Bitcoin network, formed a graph containing all off-chain channels, in which nodes and lines represent participants and off-chain channels respectively. Consequently, participants without mutually shared off-chain channels are still able to make transactions by finding a path connecting different participants between them.

However, several fundamental functionalities and limitations are still left to be developed and improved. One limitation is the malleability of payment channels. In particular, the channel with fixed capacity does not support dynamic deposit and withdrawl, which will result in extra fees for channel management and other possible inconveniences in many scenarios.

Wanting to improve payment channels’ malleability, we address both dynamic channel deposit (a.k.a. splice-in) and withdrawl (a.k.a. splice-out) problems. We have constructed a case to demonstrate that at least one on-chain broadcast is necessary for splice-in and splice-out. We then propose two approaches for splice-in and splice-out that only result in one on-chain broadcast, reducing the number of necessary broadcasts to the minimum.