\section{Distributed Detection}
\label{sec:distributed}

In this section, we elaborate how we perform heavy key detection in a
distributed architecture via LD-Sketch. 
%Our primary goal is to provide both performance and accuracy guarantees for
%our detection process.

%We first consider two possible distributed schemes for LD-Sketch, and discuss
%their limitations.  One reduces the complexity but compromises the accuracy
%goal.  The other guarantees high accuracy as its local detection algorithm,
%but introduces serious complexity problem in the distributed aperture.
%
%The first scheme is to monitor each local stream in the remote site
%The detection can be done directly in the remote site,
%or in one or more workers by forwarding the local streams to them,
%or even in a hierarchy topology as \cite{Manjhi2005}.
%This scheme offloads the workload of detecting the global stream into remote sites and workers
%and reduces the space and time complexity in each remote site and worker.
%
%However, it fails to guarantee high detection accuracy.
%This scheme implicitly assumes that the total value of a key is evenly distributed in the local streams.
%If the distribution of a key is highly skewed, the scheme may cause a terrible error rate.
%
%{\bf Scheme 2.}
%The second scheme is to maintain a sketch for the local stream.
%The sketches will be merged into one for the global stream.
%The detection are performed based on the global sketch.
%Many variants exist with different deployment of the sketches in the remote sites and workers.
%
%There are two drawbacks in this scheme.
%One is due to the local detection algorithm of LD-Sketch,
%which contains a dictionary in each of its bucket.
%However, to our best knowledge, it remains an open problem whether the dictionaries of a bucket in different local sketches can be merged.
%The other drawback is more general for all sketch-based algorithms.
%To merge the sketches, all of them must have identical size as the one for global stream.
%As shown in Table \ref{tab:compare}, the size of global stream typically determined
%by the total value, the approximation parameter and the error rate may be serious huge.
%Moreover, the detection procedure on the merged sketch may overload the workers.
%In summary, this scheme fails to optimize the complexity and compromises the resource constrains in the remote sites and workers.
%
%\subsection{Distributed Scheme in LD-Sketch}
%\label{sec:distribute_ld}

\subsection{Design}

We design our distributed scheme as follows.  We deploy an LD-Sketch in
each of the $q$ workers.  Then we partition the data stream in each of the $p$
remote sites using a two-step approach.  First, for a data item $(x,v_x)$, a
remote
site hashes key $x$ to a set of $d$ ($1\le d\le q$) workers for some design
parameter $d$.  This selection is deterministic in the sense that the same set
of $d$ workers will always be selected for all data items with the same key
$x$.  Second, the remote site uniformly selects one of the $d$ workers and
forwards it the data item.  The partitioning functions in both steps are
identical in all remote sites.  Since each of the $d$ workers receives on
average $1/d$ of the total value for each key, we set the heavy key threshold
to be $(1-\gamma)\frac{\phi}{d}$ for some approximation parameter 
$0\le \gamma <1$.
Finally, if a key is reported as a heavy key by all the $d$ workers, then we
report the key as a heavy key in our final results.

The design parameter $d$ in our distributed scheme determines the accuracy of
heavy key detection.  
%By increasing $d$, we divide the workload of updating data items of the same
%key across multiple workers, thereby improving the performance of both update
%and detection procedures.  For accuracy, 
As $d$ increases, the false positive rate decreases until some turning point
is reached.  We also introduce a small false negative rate, since some of the
$d$ workers may receive less than $(1-\gamma)\frac{\phi}{d}$ of the total
value if the partition is not perfectly even.  We study the impact of $d$ on
the accuracy in the next subsection.

\subsection{Analysis}
\label{subsec:dist_analysis}

%The complexity in the remote site is immediate.  A remote site simply invokes
%a constant number of hash functions for each arriving data item.  So both the
%time and space complexity in the remote sites is constant.

%For the analysis of workers running detection module,
%we assume that the random selection function and hash function are perfectly random.
%With this assumption, the expectation of total value of the sum-stream received
%by a worker is $\frac{U}{q}$.
%Additionally, the threshold is each worker is set to $\frac{\phi}{d}$ as presented above.

We conduct theoretical analysis on LD-Sketch in a distributed setting.  Our
results are derived from those in Section~\ref{sec:local} by choosing new
parameters for the distributed scheme.

We first consider the space and time complexity. Suppose that each LD-Sketch
has $r$ rows and $w$ buckets.  For the total sum $U$ and the threshold $\phi$
in local detection, we replace them with $\frac{U}{q}$ and
$(1-\gamma)\frac{\phi}{d}$, respectively.  We assume that the worker selection
for each data item is independent.  By similar arguments in
Lemma~\ref{lemma:bucket_length}, the expectation of the term $U^2$ in the
space and time complexity can be replaced with $O(\frac{U^2}{q^2})$, assuming
that $q$ is far smaller than the number of available keys $n$.  Using the
above arguments, the space complexity is $O(r(w+\frac{U^2}{wT^2q^2}))$.  The
time complexity of the update procedure is $O(r(1+\frac{U^2}{w^2T^2q^2}))$, and
that of the detection procedure is $O(r(w+\frac{U^2}{wT^2q^2}))$.  We
set $T=(1-\gamma)\frac{\phi}{d}$ for heavy hitter detection and
$T=(1-\gamma)\frac{\epsilon\phi}{d}$ for heavy changer detection.  Since $1\le
d \le q$ and $\gamma$ is typically small, we actually reduce both the space
and time complexity of local detection.

For the false positive rate, it can be derived from
Lemmas~\ref{lemma:hitter_fp} and \ref{lemma:changer_fp} by replacing $U$ and
$\phi$ with $\frac{U}{q}$ and $(1-\gamma)\frac{\phi}{d}$, respectively.
Note that the distributed scheme further reduces the false positive rate
since a non-heavy key needs to be reported by all $d$ workers.  Thus, the
corresponding false positive rate is at most 
$(\frac{dU}{qw(1-\gamma)\phi})^{rd}$ for heavy hitter detection and at most
$(\frac{2dU}{qw(1-\gamma)\epsilon\phi})^{rd}$ for heavy changer detection.  

We illustrate how the false positive rate varies with different values of $d$. 
As a case study, we pick $\phi=0.001U$, so there are at most $H=\frac{U}{\phi}
= 1000$ heavy hitters and $H=\frac{2U}{\phi} = 2000$ heavy changers (see
Section~\ref{subsec:heavy}).  We fix $\gamma=0$, $\epsilon=0.5$, $r=5$, and 
$w = 4500$.  Figures~\ref{fig:vary_d}(a) and \ref{fig:vary_d}(b) show the
upper bounds of the false positive rate for heavy hitter and heavy changer
detection, respectively.  Note that the upper bound of the false positive rate
can be viewed as the function $(cd)^{rd}$, for some constant $c$.  As $d$
increases, if $\log(cd) < -1$, the upper bound of the false positive rate
decreases; if $\log(cd) > -1$, the upper bound increases exponentially, as
shown in Figure~\ref{fig:vary_d}(b). 

\begin{figure}[!t]
\centering
\begin{tabular}{c@{\ }c}
\includegraphics[width=1.6in]{figs/dist/var_d/hitter_fp.eps} &
\includegraphics[width=1.6in]{figs/dist/var_d/changer_fp.eps} \\
{\small (a) Heavy Hitter} &
{\small (b) Heavy Changer} \\
\end{tabular}
\caption{False positive rate of distributed scheme versus $d$.}
\label{fig:vary_d}
\end{figure}


For the false negative rate, Lemma~\ref{lemma:dist_fn} shows that it can be
bounded in a distributed setting.
\begin{lemma}
For a heavy key $x$, it will be missed at probability at most $1-(1-e^{-\frac{\phi}{2d}\gamma^2})^d$.
\label{lemma:dist_fn}
\end{lemma}
\begin{proof}
Let $X$ be the random variable of the sum of values of the heavy key $x$
received by a worker. We first consider heavy hitter detection.  Note that
$E(X) = \frac{S(x)}{d} \ge \frac{\phi}{d}$.  By Chernoff bound, $Pr\{X <
(1-\gamma)\frac{\phi}{d}\} <
e^{-\frac{E(X)}{2}(1-(1-\gamma)\frac{d\phi}{E(X)})^2} \le
e^{-\frac{\phi}{2d}\gamma^2}$.  This is also the probability of missing the
key in a single worker.  Thus, the probability of missing the key in at least
one worker is at most $1-(1-e^{-\frac{\phi}{2d}\gamma^2})^d$.

For heavy changer detection, we denote $X$ as the difference of the heavy key 
$x$.  Since $E(X) \ge \frac{\phi}{d}$.  We can apply the same results as
above.  
\end{proof}

Finally, we propose the parameter selection for the distributed scheme.
The selection involves $H$, $\phi$, $\epsilon$, $\delta$, and $\gamma$ as
inputs.  For heavy hitter detection, LD-Sketch selects
$w=\frac{2dH}{(1-\gamma)q}$, $r=\log{\frac{1}{d\delta}}$, and
$T=\frac{(1-\gamma)\phi}{d}$; for heavy changer detection, it selects
$w=\frac{2dH}{(1-\gamma)q\epsilon}$, $r=\log{\frac{1}{d\delta}}$, and
$T=\frac{(1-\gamma)\epsilon\phi}{d}$.  With the selected parameters, the
following theorems summarize the error bounds, space complexity, and time
complexity of the distributed scheme of LD-Sketch.
%
\begin{theorem}
Consider an LD-Sketch with $w=\frac{2dH}{(1-\gamma)q}$,
$r=\log{\frac{1}{d\delta}}$, and $T=\frac{(1-\gamma)\phi}{d}$.
A heavy hitter is missed with probability at most
$1-(1-e^{-\frac{\phi}{2d}\gamma^2})^d$, while a non-heavy hitter with the true
sum less than $(1-\gamma)\phi$ is reported with probability at most $\delta$.
The expected space complexity is
$O(\frac{dH}{q(1-\gamma)}\log{\frac{1}{d\delta}})$.
The expected time complexity of updating a data item is
$O(\log{\frac{1}{d\delta}})$, and that of detection is
$O(\frac{dH}{q(1-\gamma)}\log{\frac{1}{d\delta}})$.
\end{theorem}

\begin{theorem}
Consider two LD-Sketches with $w=\frac{2dH}{(1-\gamma)q\epsilon}$,
$r=\log{\frac{1}{d\delta}}$, and $T=\frac{(1-\gamma)\epsilon\phi}{d}$.
A heavy changer is missed with probability at most
$1-(1-e^{-\frac{\phi}{2d}\gamma^2})^d$, while a non-heavy changer with the true
difference less than $(1-\gamma)(1-\epsilon)\phi$ is reported with probability
at most $\delta$.  The expected space complexity is
$O(\frac{dH}{q(1-\gamma)\epsilon}\log{\frac{1}{d\delta}})$.
The expected time complexity of updating a data item is
$\log{\frac{1}{d\delta}}$, and that of detection is
$O(\frac{dH}{q(1-\gamma)\epsilon}\log{\frac{1}{d\delta}})$.
\end{theorem}
